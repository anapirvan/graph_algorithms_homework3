\subsection*{a)}
Let $T$ be a $\beta$-tree of order $n > 2$.
The condition $\Delta(H) = \delta(H) = 1$ for the partial graph $H$ implies that $H$ is a perfect matching of $T$. Thus, each vertex $v \in V$ is incident to exactly one edge from $E(H) \subseteq E(T)$. \
Let $P = (v_0, v_1, v_2, \dots, v_k)$ be a longest path in $T$. The endpoints $v_0$ and $v_k$ are vertices of degree 1 (leaves), so we have $d_T(v_0) = 1$ and $d_T(v_k) = 1$. Since $v_0$ has only one neighbor ($v_1$), in order to be covered by the matching $H$, the edge $(v_0, v_1)$ must belong to $H$. Similarly, the edge $(v_{k-1}, v_k)$ must belong to $H$. \
If $d_T(v_1) > 2$, then there would exist another neighbor $x \in V \setminus {v_0, v_2}$. We have two cases:
\begin{itemize}
\item $x$ is a leaf, which means it cannot be included in $H$, because $v_1$, its only neighbor, is already matched with $v_0$, contradicting the $\beta$-tree property.
\item $x$ is not a leaf, which implies that $x$ has at least one more neighbor $y \neq v_1$. One can construct in $T$ a path $P'$ that starts from a leaf of the subtree containing $x$, passes through $x$ and $v_1$, and then continues along the path $P$ until reaching $v_k$. The path $P'$ is longer than $P$, contradicting the fact that $P$ is a longest path in $T$.
\end{itemize}
Therefore, the only neighbors of $v_1$ in $T$ are $v_0$ and $v_2$, hence $d_T(v_1) = 2$. Similarly, $d_T(v_{k-1}) = 2$. Thus, there exist 2 vertices of degree 1 ($v_0$ and $v_k$) adjacent to vertices of degree 2 ($v_1$ and $v_{k-1}$).

\subsection*{b)}
From the previous point, we know that in any $\beta$-tree $T$ there exists a pair of adjacent vertices $v_0$ and $v_1$ with $d_T(v_0) = 1$ and $d_T(v_1) = 2$. \
Let $\overline{T} = T \setminus {v_0, v_1}$ be the subgraph induced by $V(T) \setminus {v_0, v_1}$. Removing these two vertices does not disconnect the graph and does not introduce cycles, so $\overline{T}$ is a tree of order $|V(\overline{T})| > 2$. \
Let $H$ be the perfect matching of $T$. Since $d_T(v_0)=1$ and $v_0$ must have degree 1 in $H$, we have $(v_0, v_1) \in E(H)$. Let $\overline{H}$ be the restriction of $H$ to $\overline{T}$. The edge set is $E(\overline{H}) = E(H) \setminus {(v_0, v_1)}$. Since $v_0$ and $v_1$ were removed together with the edge matching them, all the remaining vertices in $V(\overline{T})$ are still matched, therefore $\overline{H}$ is a perfect matching in $\overline{T}$ and $\overline{T}$ is a $\beta$-tree. \
Therefore, $T$ is obtained from $\overline{T}$ by adding the vertices $v_0$ and $v_1$. The resulting degrees in $T$ are $d_T(v_0)=1$ and $d_T(v_1)=2$.

\subsection*{c)}
Starting from a $\beta$-tree $\overline{T}$, we may choose any vertex $u \in V(\overline{T})$ to which we attach the vertex $v_1$ that is adjacent to $v_0$ ($v_0, v_1 \notin V(\overline{T})$), obtaining a new graph $T$ in which $d_T(v_0) = 1$ and $d_T(v_1) = 2$. This construction does not introduce cycles and does not disconnect the graph, so $T$ is a tree. Moreover, the two added vertices extend the matching with the edge between them, so $T$ will have a perfect matching. \

\subsection*{d)}
An efficient algorithm for identifying the maximum-cardinality matching in a $\beta$-tree is Greedy-type and is based on a post-order traversal of the tree. When a node is reached, if it is unmatched, we match it with its parent, and the edge between the two nodes is added to the matching. This decision ensures local optimality because a leaf can only be matched with its parent. The algorithm has complexity $O(n)$, since each edge and each node is traversed exactly once.