\vspace{-2ex}
\subsection*{1. Max-Flow Min-Cut Theorem:}
The maximum value of a flow in the network $R=(G,s,t,c)$ is equal to the minimum capacity of a cut in $R$.

\subsection*{2. Description of the algorithm:}
\begin{itemize}
\item \textbf{Step I.} We compute the maximum flow of the given network $R=(G,s,t,c)$ and we denote the value of the maximum flow by $f_{max}$.
\item \textbf{Step II.} We construct the network $R_0$ in which the capacity of the edge $e_0$ (where $e_0=u_0v_0 \in E(G)$) will be $c(e_0)+\alpha$ (where $\alpha = \min{|c(e')-c(e'')| : e', e'' \in E(G), c(e') \ne c(e'')}$).
\item \textbf{Step III.} We compute the maximum flow of the network $R_0$ and we denote the value of this flow by $f'{max}$.
\item \textbf{Step IV.}
\begin{itemize}
\item If $f'{max} > f_{max}$ then $e_0 \in E(S,T)$ for any minimum cut $(S,T)$ of $R$, the answer is $true$.
\item If $f'{max} = f{max}$ then the answer is $false$.
\end{itemize}
\end{itemize}

\subsection*{3. Proof of Correctness:}
We need to prove that $f'{max}>f_{max} \iff \forall (S,T)$ the minimum cut in $R$, we have $e_0 \in E(S,T).$

\begin{itemize}
\item \textbf{“$\Leftarrow$” direction.}
Assume that $e_0$ belongs to all minimum cuts in $R$. Let $C$ be the capacity of the minimum cut in $R$. According to the Max-Flow Min-Cut Theorem, $f_{max}=C$. Because $e_0$ is part of all the cuts of capacity $C$, increasing the capacity of $e_0$ by $\alpha$ will increase the capacity of all these cuts to $C+\alpha$. Any other cut $(S',T')$ that does not contain the edge $e_0$ already has (by hypothesis) strictly larger capacity than $C$ (because it is not a minimum cut; otherwise it would contain $e_0$). Therefore, the minimum capacity of a cut in $R_0$ will be strictly greater than $C$. According to the Max-Flow Min-Cut Theorem, the new maximum flow $f'{max}>C=f{max}$.

\item \textbf{“$\Rightarrow$” direction.}
Assume, by contradiction, that $f'{max}>f{max}$ and that there exists a minimum cut $(S',T')$ in $R$ which does not contain the edge $e_0$. The capacity of the cut $(S',T')$ is equal to $f_{max}$. In the network $R_0$, the capacity of the edge $e_0$ increases, but the capacities of the other edges remain unchanged. Because $e_0 \notin E(S',T')$, the capacity of this cut will remain unchanged in $R_0$. The maximum flow is at most the capacity of any cut. Therefore $f'{max} \le C{R_0}(S',T')=C_R(S',T')=f_{max}$, i.e. $f'{max} \le f{max}$, which contradicts the assumption $f'{max} > f{max}$. Therefore the assumption was false, $e_0$ must belong to all minimum cuts.
\end{itemize}